%
% File: abstract.tex
% Author: Tengxiang Li
%
\chapter{Abstract}
\begin{SingleSpace}
In the modern software development lifecycle, developers are increasingly tasked with the critical responsibility of writing secure code. However, the landscape of security tools available is often fragmented, presenting a collection of disparate and complex utilities that are difficult to integrate into a seamless workflow. This fragmentation creates a significant barrier to the widespread adoption of secure coding practices from the outset of a project.

This project addresses this gap by designing and building a unified, developer-centric security toolkit. The core of this project is a command-line interface (CLI) application, architected in Python for its extensive libraries and ease of development. The toolkit is an extensible framework that integrates two distinct scanning engines into a single CLI-driven workflow. It features a \textbf{Custom Engine} with modules for static analysis of Python source code, dependency scanning, and configuration review. As a complementary feature, it also integrates the powerful, general-purpose \textbf{Semgrep Engine}, allowing the user to choose the best approach for their needs. The tool is guided by the principles of the OWASP Top 10, focusing on flaws such as injection, broken access control, and insecure design patterns.

A key component of this project is a companion Visual Studio Code (VS Code) extension. This extension bridges the gap between security analysis and development by providing real-time, contextual feedback directly within the developer's editor. By doing so, the project aims to transform security from a separate, often-delayed process into an integral, immediate part of the coding experience.

The toolkit's effectiveness will be evaluated based on its functionality, performance, and usability. The evaluation will assess the toolkit's detection capabilities against curated test cases of insecure code and misconfigured environments. Its performance will be benchmarked against established, single-purpose security tools to provide a comparative analysis. This project aims to contribute a practical and powerful solution that empowers developers to build more secure software by embedding security seamlessly into their daily development habits.
\end{SingleSpace}
% \clearpage