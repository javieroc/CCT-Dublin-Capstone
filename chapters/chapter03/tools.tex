% Chapter 3: Existing Tools for CI/CD
\chapter{Existing Tools for CI/CD}
\label{chap:tools}
\setlength{\parskip}{1em}

This chapter provides an overview of existing tools used for identifying security vulnerabilities in software, with a focus on those that are well-suited for integration into a CI/CD pipeline. We will focus on Static Application Security Testing (SAST) and Software Composition Analysis (SCA) tools, as these are the most relevant for a source code analysis tool.

\section{Static Application Security Testing (SAST)}
SAST tools analyze source code, byte code, or binary code for security vulnerabilities without executing the application. This makes them ideal for use in a CI/CD pipeline, as they can be run before the code is deployed.

\begin{itemize}
    \item \textbf{Semgrep:} An open-source, multi-language static analysis tool that supports custom rules written in YAML. It is fast and suitable for CI/CD pipelines.
    \begin{verbatim}
# Example: Find hardcoded secrets in Python code
semgrep --config="p/ci" --error .
    \end{verbatim}
    \item \textbf{Bandit:} A tool designed for finding common security issues in Python code. It works by analyzing the Abstract Syntax Tree (AST) of the code.
    \begin{verbatim}
# Example: Scan a Python file for vulnerabilities
bandit -r examples/hardcoded_password.py
    \end{verbatim}
\end{itemize}

\subsection{Feasible Features for Our Tool}
Given the time constraints, we will focus on implementing a few key features from these tools:
\begin{itemize}
    \item \textbf{Custom Rules:} We will implement a simple rule engine that can be used to define custom rules for our tool. This will allow us to tailor the tool to our specific needs.
    \item \textbf{AST Analysis:} We will use an existing library to parse the source code into an Abstract Syntax Tree (AST). This will allow us to perform more complex analysis of the code.
    \item \textbf{JSON Output:} We will provide a JSON output option for our tool. This will make it easy to integrate our tool with other tools, such as a CI/CD pipeline.
\end{itemize}

\section{Software Composition Analysis (SCA)}
SCA tools identify and track open-source components used in an application. They check for known vulnerabilities in these components and can also enforce license policies.

\begin{itemize}
    \item \textbf{OWASP Dependency-Check:} An open-source SCA tool that identifies project dependencies and checks for known, publicly disclosed vulnerabilities. It can be used as a command-line tool or as a plugin for build tools like Maven and Gradle.
    \begin{minted}{bash}
# Example: Scan a project for vulnerable dependencies
dependency-check.sh --scan /path/to/project --format JSON
    \end{minted}
\end{itemize}

\subsection{Feasible Features for Our Tool}
We will implement a simple SCA feature that can be used to identify the dependencies of a project and check for known vulnerabilities. We will use an existing library to parse the project's dependency file (e.g., `requirements.txt` for Python) and then use an online vulnerability database to check for known vulnerabilities.

\section{Differentiation of the Developed Tool}

While there are mature static analysis frameworks such as \textbf{Semgrep}, my tool is not intended to replicate their broad functionality. Instead, it has been designed with a focused scope and unique characteristics that set it apart:

\subsection{Specialization in OWASP Top 10 Vulnerabilities}
Unlike generic static analyzers, this tool is purpose-built to target the \textbf{OWASP Top 10} categories. Each scanning module is mapped to a specific risk category, such as:
\begin{itemize}
    \item \textbf{Injection} $\rightarrow$ Detection of SQL and NoSQL injection patterns.
    \item \textbf{Insecure Design} $\rightarrow$ Identification of insecure use of dangerous functions (\texttt{eval}, \texttt{exec}, \texttt{pickle.loads}, etc.).
    \item \textbf{Security Misconfiguration} $\rightarrow$ Discovery of risky framework defaults (e.g., \texttt{DEBUG=True}, insecure cookies, hardcoded secret keys).
    \item \textbf{Cryptographic Failures} $\rightarrow$ Weak hash usage (\texttt{MD5}, \texttt{SHA1}), hardcoded cryptographic keys.
\end{itemize}

This focused approach ensures that the tool provides actionable findings directly aligned with industry-recognized security priorities, instead of overwhelming developers with thousands of generic rules.

\subsection{Emphasis on Educational Value}
The tool is not only a vulnerability scanner but also an \textbf{educational resource}. Each detected issue is accompanied by:
\begin{itemize}
    \item A clear explanation of \textbf{why} the pattern is insecure.
    \item A description of the \textbf{potential attack scenario} (e.g., how enabling debug mode can expose stack traces and sensitive environment variables).
    \item Guidance on \textbf{remediation strategies} to resolve the issue securely.
\end{itemize}

This makes the tool especially valuable for students, junior developers, and teams new to secure coding practices, as it doubles as a \textbf{teaching aid} in addition to a security scanner.

\subsection{Multi-Scanner Orchestration}
The tool integrates multiple scanning approaches into a single CLI-driven workflow:
\begin{itemize}
    \item \textbf{Static code analysis} using Python’s AST and regex for pattern detection.
    \item \textbf{Dependency scanning} via pip-audit to flag outdated or vulnerable libraries.
    \item \textbf{Configuration scanning} of common framework files (e.g., Django \texttt{settings.py}, Flask \texttt{app.py}, \texttt{.env} files) to identify misconfigurations.
\end{itemize}

By orchestrating these methods into a unified report and summary, the tool provides developers with a \textbf{consolidated view of their project’s security posture}, something not commonly found in traditional static analysis tools.

\vspace{1em}
\noindent In summary, the uniqueness of this tool lies in its \textbf{OWASP-focused scope}, its \textbf{educational emphasis}, and its ability to \textbf{combine multiple scanning techniques into one coherent solution}, making it both practical and instructional.