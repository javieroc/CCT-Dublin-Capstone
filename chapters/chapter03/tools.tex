% Chapter 3: Existing Tools
\chapter{Existing Tools}
\label{chap:tools}
\setlength{\parskip}{1em}

This chapter provides an overview of existing tools used for identifying security vulnerabilities in software. The tools are categorized based on their analysis techniques: Static Application Security Testing (SAST), Dynamic Application Security Testing (DAST), and Interactive Application Security Testing (IAST).

\section{Static Application Security Testing (SAST)}
SAST tools analyze source code, byte code, or binary code for security vulnerabilities without executing the application.

\begin{itemize}
    \item \textbf{Semgrep:} An open-source, multi-language static analysis tool that supports custom rules written in YAML. It is fast and suitable for CI/CD pipelines.
    \item \textbf{Bandit:} A tool designed for finding common security issues in Python code. It works by analyzing the Abstract Syntax Tree (AST) of the code.
    \item \textbf{Insider:} A SAST tool focused on the OWASP Top 10 vulnerabilities. It supports multiple languages, including Java, C#, JavaScript, and Python.
\end{itemize}

\section{Dynamic Application Security Testing (DAST)}
DAST tools test the running application for vulnerabilities by sending requests and analyzing the responses.

\begin{itemize}
    \item \textbf{OWASP ZAP:} An open-source web application security scanner. It can be used as a proxy to intercept and modify traffic, as well as for automated scanning.
    \item \textbf{Nikto:} An open-source web server scanner that checks for outdated software, dangerous files, and other common vulnerabilities.
\end{itemize}

\section{Interactive Application Security Testing (IAST)}
IAST tools combine elements of both SAST and DAST. They use agents to instrument the application and monitor its behavior during runtime.

\begin{itemize}
    \item \textbf{Contrast Security:} A commercial IAST tool that integrates with the application to provide real-time vulnerability detection and feedback.
    \item \textbf{VULAS (Vulnerability Assessment Tool):} An open-source, language-agnostic tool that uses a hybrid approach to identify vulnerabilities by analyzing both the source code and the runtime behavior of the application.
\end{itemize}

\section{Software Composition Analysis (SCA)}
SCA tools identify and track open-source components used in an application. They check for known vulnerabilities in these components and can also enforce license policies.

\begin{itemize}
    \item \textbf{OWASP Dependency-Check:} An open-source SCA tool that identifies project dependencies and checks for known, publicly disclosed vulnerabilities. It can be used as a command-line tool or as a plugin for build tools like Maven and Gradle.
    \item \textbf{Snyk:} A commercial SCA tool that provides vulnerability scanning for open-source dependencies and container images. It offers IDE and CI/CD integration.
\end{itemize}

\section{HTTP Header Scanners}
HTTP header scanners check the security of HTTP response headers.

\begin{itemize}
    \item \textbf{shcheck:} A command-line tool for checking HTTP security headers. It provides a simple pass/fail scoring system.
\end{itemize}

