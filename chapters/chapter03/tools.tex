% Chapter 3: Existing Tools for CI/CD
\chapter{Existing Tools for CI/CD}
\label{chap:tools}
\setlength{\parskip}{1em}

This chapter provides an overview of existing tools used for identifying security vulnerabilities in software, with a focus on those that are well-suited for integration into a CI/CD pipeline. We will focus on Static Application Security Testing (SAST) and Software Composition Analysis (SCA) tools, as these are the most relevant for a source code analysis tool.

\section{Static Application Security Testing (SAST)}
SAST tools analyze source code, byte code, or binary code for security vulnerabilities without executing the application. This makes them ideal for use in a CI/CD pipeline, as they can be run before the code is deployed.

\begin{itemize}
    \item \textbf{Semgrep:} An open-source, multi-language static analysis tool that supports custom rules written in YAML. It is fast and suitable for CI/CD pipelines.
    \begin{verbatim}
# Example: Find hardcoded secrets in Python code
semgrep --config="p/ci" --error .
    \end{verbatim}
    \item \textbf{Bandit:} A tool designed for finding common security issues in Python code. It works by analyzing the Abstract Syntax Tree (AST) of the code.
    \begin{verbatim}
# Example: Scan a Python file for vulnerabilities
bandit -r examples/hardcoded_password.py
    \end{verbatim}
\end{itemize}

\subsection{Feasible Features for Our Tool}
Given the time constraints, we will focus on implementing a few key features from these tools:
\begin{itemize}
    \item \textbf{Custom Rules:} We will implement a simple rule engine that can be used to define custom rules for our tool. This will allow us to tailor the tool to our specific needs.
    \item \textbf{AST Analysis:} We will use an existing library to parse the source code into an Abstract Syntax Tree (AST). This will allow us to perform more complex analysis of the code.
    \item \textbf{JSON Output:} We will provide a JSON output option for our tool. This will make it easy to integrate our tool with other tools, such as a CI/CD pipeline.
\end{itemize}

\section{Software Composition Analysis (SCA)}
SCA tools identify and track open-source components used in an application. They check for known vulnerabilities in these components and can also enforce license policies.

\begin{itemize}
    \item \textbf{OWASP Dependency-Check:} An open-source SCA tool that identifies project dependencies and checks for known, publicly disclosed vulnerabilities. It can be used as a command-line tool or as a plugin for build tools like Maven and Gradle.
    \begin{verbatim}
# Example: Scan a project for vulnerable dependencies
dependency-check.sh --scan /path/to/project --format JSON
    \end{verbatim}
\end{itemize}

\subsection{Feasible Features for Our Tool}
We will implement a simple SCA feature that can be used to identify the dependencies of a project and check for known vulnerabilities. We will use an existing library to parse the project's dependency file (e.g., `requirements.txt` for Python) and then use an online vulnerability database to check for known vulnerabilities.
