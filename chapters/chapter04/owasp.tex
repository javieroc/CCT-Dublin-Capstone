% Chapter 3: The OWASP Framework
\chapter{The OWASP Framework and Secure Coding Principles}
\label{chap:owasp}
\setlength{\parskip}{1em}

The Open Web Application Security Project (OWASP) provides unbiased, practical, and cost-effective information about application security. Its flagship project, the OWASP Top 10, is a globally recognized standard for developers and web application security. It represents a broad consensus about the most critical security risks to web applications.

This thesis leverages the OWASP Top 10 as a foundational guide for the security rules and checks implemented in the toolkit. The tool is not designed to cover every aspect of the OWASP Top 10, but rather to focus on vulnerabilities that can be detected through static analysis of Python source code and analysis of HTTP security headers. This chapter outlines the specific OWASP categories that the toolkit will address.

\section{A01:2021 - Broken Access Control}
Access control enforces policies such that users cannot act outside of their intended permissions. Failures typically lead to unauthorized information disclosure, modification, or destruction of all data or performing a business function outside the user's limits. 
The toolkit will contribute to detecting broken access control by scanning for common anti-patterns in Python code, such as hardcoded user roles or privileges, and by identifying insecure direct object reference (IDOR) patterns where user-controllable input is used to access resources without proper authorization checks.

\section{A02:2021 - Cryptographic Failures}
This category focuses on failures related to cryptography, which can lead to the exposure of sensitive data. Common issues include the use of weak or outdated cryptographic algorithms, improper key management, and the use of hardcoded cryptographic keys.
The static analysis module of the toolkit will scan Python code for:
\begin{itemize}
    \item Use of weak hashing algorithms (e.g., MD5, SHA1).
    \item Hardcoded secrets and cryptographic keys in the source code.
    \item Insecure implementation of cryptographic functions.
\end{itemize}

\section{A03:2021 - Injection}
Injection flaws, such as SQL, NoSQL, OS, and LDAP injection, occur when untrusted data is sent to an interpreter as part of a command or query. The attacker's hostile data can trick the interpreter into executing unintended commands or accessing data without proper authorization.
The toolkit's static analysis engine will inspect the codebase for patterns indicative of injection vulnerabilities, such as the use of un-sanitized user input in database queries or shell commands.

\section{A04:2021 - Insecure Design}
Insecure design is a broad category representing different weaknesses, expressed as "missing or ineffective control design". These are not implementation bugs, but flaws in the design of the application that can lead to vulnerabilities.
The toolkit will help identify insecure design patterns by flagging code structures that are known to be risky, such as the use of functions that are inherently insecure (e.g., `eval()` in Python) or the lack of input validation in critical parts of the application.

\section{A05:2021 - Security Misconfiguration}
Security misconfiguration can happen at any level of an application stack, including the network services, platform, web server, application server, database, frameworks, and custom code. This is often a result of insecure default configurations, incomplete or ad-hoc configurations, open cloud storage, misconfigured HTTP headers, and verbose error messages containing sensitive information.
The toolkit will address this category in two ways:
\begin{itemize}
    \item The HTTP security header scanner will check for the presence and correct configuration of security-related headers (e.g., `Strict-Transport-Security`, `X-Frame-Options`, `Content-Security-Policy`).
    \item The static analysis module will look for common misconfigurations in Python code, such as running in debug mode in a production environment.
\end{itemize}

\section{A06:2021 - Vulnerable and Outdated Components}
Modern applications are built using a large number of third-party components, such as libraries and frameworks. If a vulnerable component is used, it can undermine the security of the entire application.
The toolkit will include a Software Composition Analysis (SCA) module that scans the project's dependencies (e.g., `requirements.txt`) and checks them against a database of known vulnerabilities.

\section{A07:2021 - Identification and Authentication Failures}
This category includes vulnerabilities related to user identity and authentication. Common examples include allowing brute-force attacks, using weak password policies, and improper session management.
The static analysis engine will scan for common authentication and session management anti-patterns, such as the use of predictable session identifiers or the lack of rate limiting on authentication endpoints.

\section{A08:2021 - Software and Data Integrity Failures}
This category focuses on vulnerabilities related to software updates, critical data, and the CI/CD pipeline without verifying integrity. One of the most common examples is insecure deserialization, where untrusted data is used to abuse the logic of an application, inflict a denial-of-service (DoS) attack, or even execute arbitrary code.
The toolkit's static analysis module will scan for the use of insecure deserialization functions on untrusted data.

\section{A10:2021 - Server-Side Request Forgery (SSRF)}
SSRF flaws occur whenever a web application is fetching a remote resource without validating the user-supplied URL. It allows an attacker to coerce the application to send a crafted request to an unexpected destination, even when protected by a firewall, VPN, or another type of network access control list (ACL).
The static analysis engine will look for code patterns where user-controlled data is used to construct URLs for server-side requests, which is a primary indicator of potential SSRF vulnerabilities.